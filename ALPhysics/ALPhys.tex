\documentclass[oneside]{book}
\usepackage{../Style/Master}
\usepackage{../Style/boxes}
\usepackage{../Style/DefNoteFact}
\usepackage{../Style/QnsProof}
\usepackage{../Style/Thms}
\usepackage{newpxtext, eulerpx}

\begin{document}

\begin{titlepage}
    \frontmatter
    \begin{tikzpicture}[font=\sffamily,remember picture,overlay]
        \node[fill=Dandelion,anchor=north, minimum width=\paperwidth, minimum height=2cm] (names)
     at ([yshift=-3cm]current page.north) {
      \begin{tabular}{c}
          \\
          \color{blue} {\fontsize{24.88pt}{65pt}\selectfont Document Title}\\[2mm]
          \color{blue} {\fontsize{10pt}{10pt}\selectfont By Author / Whatever}\\
          \\
          \Large \color{blue}{Name}\\
          \\
      \end{tabular}
      };
    \end{tikzpicture}

\begingroup
\let\clearpage\relax
\vspace{5cm}
\tableofcontents
\endgroup
\mainmatter
\end{titlepage}

\raggedright
\setcounter{chapter}{1}
\chapter{Kinematics}
\begin{itemize}
    \item \textit{Distance} is defined as the total length of \emph{path} travelled.
    \item \textit{Velocity} is defined as the rate of change of displacement.
    \item \textit{Acceleration} is defined as the rate of change of velocity.
\end{itemize}
\chapter{Dynamics}
\begin{itemize}
    \item \textit{Newton's First Law of Motion} states that an object at rest will remain at rest and an object in motion will remain in motion at constant velocity in a straight line in the absence of an \emph{external} resultant force.
    \item The \textit{linear momentum} of a body is the product of its mass and velocity. The linear momentum is in the \emph{same direction} as it velocity.
    \item \textit{Newton's Second Law of Motion} states that the rate of change of momentum of a body is directly proportional to the resultant force acting on the body and occurs \emph{in the direction} of the resultant force.
    \item \textit{Newton's Third Law of Motion} states that if body A exerts a force on body B, then body B exerts a force of the \emph{same type} that is equal in magnitude and opposite in direction on body A.
    \item \textit{Impulse} is defined as the product of \emph{average} force acting on an object and the time for which the force acts.
    \item The \textit{Principle of Conservation of Linear Momentum} states that the total momentum of a system remains constant provided no \emph{external} resultant force acts on the system.
\end{itemize}
\chapter{Forces}
\begin{itemize}
    \item \textit{Hooke's Law} states that the force is directly proportional to the extension in a material if its \emph{limit of proportionality} is not exceeded.
    \item The \textit{centre of gravity} of an object is the point at which the entire weight of a body may be considered to act.
    \item The \textit{moment} of a force is equal to the product of the force and the \emph{perpendicular} distance of the \emph{line of action} of the force from the pivot. It is also the turning effect of a force.
    \item \textit{Torque of a couple} is defined as the product of one of the forces and the \emph{perpendicular} distance between the \emph{lines of action} of the forces.
    \item The \textit{Principle of Moments} states that if a body is in equilibrium, the sum of all the clockwise moments about \emph{any axis} must be equal to the sum of anticlockwise moments about the \emph{same axis}.
    \item \textit{Density} is defined as the mass per unit volume of a substance.
    \item \textit{Pressure} is defined as force per unit area, where the force is \emph{acting perpendicularly} to the area.
    \item Deriving \(p=\rho gh\):
    \begin{enumerate}
        \item Consider a point at a depth \(h\) below the surface of a liquid of density \(\rho\). 
        \item The force \(F\) acting perpendicularly on a surface area \(A\) at depth \(h\) is due to the weight of the liquid column above \(A\) to give pressure \(p\). Thus, \(p=\frac{F}{A}=\frac{mg}{A}=\frac{\rho Ah}{g}=\rho gh\).
    \end{enumerate}
    \item \textit{Upthrust} is the upward force exerted by a fluid on a body immersed in the fluid (due to pressure difference in the fluid).
    \item \textit{The origin of upthrust:} Upthrust is a result of the pressure difference between top and bottom surfaces of the body, resulting in a net upwards force being exerted on the body by the third medium in which the body is located.
    \item \textit{Archimedes' Principle} states that when a body is totally or partially immersed in a fluid, it experiences an upward force (upthrust) equal to the weight of fluid displaced.
    \item \textit{The Principle of Floatation} states that, for any object floating in \emph{equilibrium}, the upthrust is equal to the weight of the object.
\end{itemize}
\chapter{Work, Energy, and Power}
\begin{itemize}
    \item \textit{Work done} is defined as the product of a force and the displacement in the direction of the force.
    \item \textit{One joule of work} is defined as the work done by a force of 1 Newton when its \emph{point of application} moves through a distance of 1 metre in the direction of the force.
    \item \textit{Energy} is defined as the ability to do work.
    \item \textit{The Principle of Conservation of Energy} states that energy can neither be created or destroyed in \emph{any process}. It can be transformed from one form to another, and transferred from one body to another.
    \item Deriving \(E_k=\frac{1}{2}mv^2\): 
    \begin{enumerate}
        \item Consider a constant horizontal applied force \(F\) acting on an object of mass \(m\) travelling with initial velocity \(u\) to reach a final velocity \(v\) over a displacement \(s\). 
        \item For uniform acceleration, \(v^2=u^2+2as\) so \(as=\frac{1}{2}(v^2-u^2)\). Combined with Newton's Second Law, \(W=Fs=mas=\frac{1}{2}mv^2-\frac{1}{2}mu^2\). When the object starts from rest, \(u=0\). 
        \item By conservation of energy,\emph{ the work done by force \(F\) must be converted into the kinetic energy \(E_k\) of the object}. Hence, \(E_k=W=\frac{1}{2}mv^2-\frac{1}{2}m(0)^2=\frac{1}{2}mv^2\).
    \end{enumerate}
    \item The \textit{Work-Energy Theorem} states that the net work done by \emph{external} forces acting on a particle is equal to the change in kinetic energy of the particle.
    \item Deriving \(E_p=mgh\):
    \begin{enumerate}
        \item Consider an object from the Earth's surface --- which is taken as the reference for zero gravitational potential energy --- raised up by a \emph{constant force \(F\) equal to and opposite to the weight \(mg\)} of the object such that the object moves up at \emph{constant velocity} to a height \(h_2\). 
        \item Thus, the object moves at constant speed so \(\Delta E_k=0\). Therefore, 
        \begin{align*}
            \Delta E_p&= W\\
            E_p-0&=Fs\\
            E_p&=mgh.
        \end{align*}
        Where \(E_p\) is the gravitational potential energy at height \(h\) above the Earth's surface.
    \end{enumerate}
    \item Know how to \(\Delta E_p=\frac{1}{2}kx^2\) from area under graph.
    \item \textit{Power} is defined as the rate of doing work.
    \item Derive \(P=Fv\): \(P=\frac{\text{d}W}{\text{d}t}=\frac{F\text{d}s}{\text{d}t}=Fv\).
\end{itemize}
\chapter{Temperature and Ideal Gases}
\begin{itemize}
    \item The \textit{Zeroth Law of Thermodynamics} If bodies \(A\) and \(B\) are separately in thermal equilibrium with body \(C\), then bodies \(A\) and \(B\) are in thermal equilibrium with each other.
    \item \textit{One mole} is defined as the amount of substance that contains as many elementary particles as there are atoms in 0.012kg of carbon-12.
    \item \textit{Avogadro's Constant \(N_A\)} is the number of atoms in 0.012kg of carbon-12.
    \item ~\\[-3mm]
    \begin{tabular}{|Sc|Sl|}
        \hline
        & Assumptions of the Kinetic Theory of Gases\\
        \hline
        \textbf{M} & 
        \begin{tabular}{@{}Sl@{}}
            The molecules of the gas are in \emph{rapid} and \emph{random} motion.
          \end{tabular}\\
        \hline
        \textbf{A} & 
        \begin{tabular}{@{}Sl@{}}
            There are \emph{no intermolecular} attractive forces.
        \end{tabular}\\
        \hline
        \textbf{N} & 
        \begin{tabular}{@{}Sl@{}}
        Any gas consists of a \emph{very large number} of molecules.
        \end{tabular}\\
        \hline
        \textbf{T} & 
        \begin{tabular}{@{}Sl@{}}
        The duration of collisions is negligible compared\\ to the time interval between collisions.
        \end{tabular}\\
        \hline
        \textbf{E} & 
        \begin{tabular}{@{}Sl@{}}
        The collisions between gas molecules, and between\\ gas molecules and the container walls are \emph{perfectly elastic}.
        \end{tabular}\\
        \hline
        \textbf{V} & 
        \begin{tabular}{@{}Sl@{}}
        The volume of the gas molecules themselves is negligible \\compared to the volume of the container.
        \end{tabular}\\
        \hline
    \end{tabular}
    \item Deriving \(p=\frac{1}{3}\frac{Nm}{V}\langle c^2 \rangle\):
    \begin{enumerate}
        \item Consider a cubic container of side \(l\) containing \(N\) molecules, each of mass m.
        \item Change in momentum due to \emph{elastic} collision between wall and molecule\(\text{}=2mc_x\)
        \item Time interval between collisions, \(\Delta t=\frac{2l}{c_x}\).
        \item By Newton's 2nd Law, \(F=\frac{2mc_x}{\frac{2l}{c_x}}=\frac{mc_x^2}{l}\).
        \item Since \(A=l^2\), Pressure due to 1 particle, \(p=\frac{mc_x^2}{l^3}=\frac{mc_x^2}{V}\).
        \item Pressure due to \(N\) particles, \(p_N=\frac{Nmc_x^2}{V}\).
        \item By Pythagoras'Theorem, \(c^2=c_x^2+c_y^2+c_z^2\). The average speed in the \(x\), \(y\), and \(z\) directions can be taken to be \(c_x=c_y=c_z\) so \(c^2=3c_x^2\). Now, \(p_N=\frac{Nm\langle \frac{1}{3}c^2\rangle}{V}=\frac{1}{3}\frac{Nm\langle c^2 \rangle}{V}\).
    \end{enumerate}
\end{itemize}
\chapter{First Law of Thermodynamics}
\begin{itemize}
    \item The \textit{heat capacity} of a body is defined as the amount of thermal energy required to raise its temperature by one Kelvin / degree Celsius.
    \item The specific \textit{heat capacity} of a body is defined as the amount of thermal energy required to raise the temperature of one unit mass of the substance by one Kelvin / degree Celsius.
    \item The \textit{specific latent heat} of a body is defined as the thermal energy required to change \emph{phase} of one unit mass of a substance, \emph{without a change in temperature}.
    \item \textit{Internal energy} of a system is a sum of \emph{random distribution} of kinetic and potential energy \emph{associated with the molecules} of the system.
    \item The \textit{First Law of Thermodynamics} states that the \emph{increase} in internal energy of a closed system is the \emph{sum} of heat \emph{supplied} to the system and the work done \emph{on} the system. 
\end{itemize}
\chapter{Circular Motion}
\begin{itemize}
    \item \textit{Angular displacement} is the angle through which an object turns \emph{with respect to the centre} of the circular path.
    \item \textit{The radian} is defined as the angle \emph{subtended} at the \emph{centre} of a circle by an \emph{arc} of length equal to the radius of the circle. 
    \item \textit{Angular velocity} is the rate of change of angular displacement.
\end{itemize}
\begin{itemize}[label=\(\square\)]
    \item \begin{tabular}{|Sc|Sc|Sc|Sc|Sc|}
        \hline
            \(\begin{aligned}
                \omega=\frac{2\pi}{T}=2\pi f
            \end{aligned}\)&
            \(\begin{aligned}
                v=r\omega
            \end{aligned}\)&
            \(\begin{aligned}
                a_c=\frac{v^2}{r}=r\omega^2=v\omega
            \end{aligned}\)&
            \(\begin{aligned}
                F_c=ma_c
            \end{aligned}\)
        \\
        \hline
    \end{tabular}
    \item Common formulae: \(\theta=\tan^{-1}\left(\frac{v^2}{rg}\right)\), \(v=\sqrt{rg}\).
    \item Water in bucket at top position: \(F_c=N+W\) (where \(N\geq 0\)) so \(\omega>\sqrt{\frac{g}{r}}\).
    \item Need to write ``Centripetal force is provided by \underline{\hspace{1cm}}'' 
\end{itemize}
\chapter{Gravitational Fields}
\begin{itemize}
    \item \textit{Newton's Law of Gravitation} states that the force of attraction between any two point masses is directly proportional to the product of their masses and inversely proportional to the square of their separation.
    \item A \textit{gravitational field} is a region in space where mass experiences a gravitational force acting on it.
    \item Gravitational field strength at a point is defined as the gravitational force per unit mass acting on a small mass placed at that point
    \item The \textit{gravitational potential energy} of a mass at a point is defined as the work done by an \emph{external agent} in bringing the mass \emph{from infinity} to that point (without any change in kinetic energy).
    \item \textit{Gravitational potential} at a point is defined as the work done per unit mass by an \emph{external agent} in bringing a mass \emph{from infinity} to that point (without a change in kinetic energy).
    \item Escape velocity is the \emph{minimum} velocity a mass needs to be projected from the \emph{surface} of the moon in order to have sufficient kinetic energy to overcome the gravitational field it experiences and \emph{move to infinity}.
\end{itemize}
\begin{itemize}[label=\(\square\)]
    \item \[
\begin{tikzcd}[row sep=large, column sep=large]
     U_G=-\frac{GMm}{r} \arrow{r}{-\frac{\text{d}}{\text{d}r}} \arrow[swap]{d}{\frac{1}{m}} & F_G=-\frac{GMm}{r^2} \arrow[swap]{d}{\frac{1}{m}} \\
     \phi=-\frac{GM}{r} \arrow{r}{-\frac{\text{d}}{\text{d}r}} & g=-\frac{GM}{r^2}\\
\end{tikzcd}
\]
\item \(U_G=m\phi\) \& \(\Delta U_G=m\Delta\phi\).
\item \(U_G\) is negative because infinity is taken as the reference point for zero potential energy.
\item Gravitational force provides the centripetal force:
\begin{align*}
    F_G&=F_c\\
    \frac{GMm}{r^2}&=mr\omega^2=mr\left(\frac{2\pi}{T}\right)^2\\
    T^2&=\frac{4\pi^2}{GM}r^3\\
    T^2 &\propto r^3
\end{align*}
\item Gravitational force provides the centripetal force:
\begin{alignat*}{2}
    && F_G&=F_c\\
    &\text{For \(A\):}& \hspace{1cm} \frac{Gm_Am_B}{(r_A+r_B)^2}&=m_Ar_A\omega^2\\
    &\text{For \(B\):}& \frac{Gm_Am_B}{(r_A+r_B)^2}&=m_Br_B\omega^2
\end{alignat*}
The centre of mass of the system is at point \(P\) where 
\[m_Ar_A=m_Br_B\]
such that both stars have the same angular velocity \(\omega\).
\item Escape velocity \(v_\text{min}=\sqrt{\frac{2GM}{r}}\) (where Min \(E_k\) needed is the gain in \(E_p\) to reach infinity).
\item Geostationary orbit facts:
\begin{enumerate}
    \item Only one such orbit at a \emph{fixed} distance of \(4.2\times10^7\text{m}\) from Earth's centre,
    \item Orbital period of 24 hours,
    \item Satellite's plane of orbit coincides with the equatorial plane of the Earth,
    \item Orbits West to East (in the same direction as Earth's rotation). 
\end{enumerate}
\item Equipotential lines are not equally spaced because the gravitational field strength is not constant but decreases as one goes away from the Earth.
\end{itemize}
\chapter{Oscillations}
\begin{itemize}
    \item \textit{Simple harmonic motion} is defined as the motion of a body whose acceleration is directly proportional to its displacement from a fixed point (equilibrium position) and is always directed towards that fixed point.
    \item A \textit{freely oscillating} system oscillates at its own \textit{natural frequency} without \emph{external} influences other than the \emph{initial impulse when displaced} from its equilibrium position, with \emph{no dissipation} of energy.
    \item \textit{Damped oscillations} are oscillations in which the amplitude diminishes with time as a result of \emph{dissipative forces} that reduce the total energy of the oscillations.
    \item A system is in \textit{forced oscillations} when it is forced to oscillate at a frequency other than the natural frequency by a \emph{periodic external} force.
    \item \textit{Resonance} is a phenomenon that occurs when the frequency at which an object is being made to vibrate (the forced frequency of vibration) is equal equal to its natural frequency of vibration.
\end{itemize}
\begin{itemize}[label=\(\square\)]
    \item \begin{tabular}{|Sc|Sc|Sc|Sc|Sc|}
        \hline
        \multirow{2}{*}[-1em]{\(\begin{aligned}
            v=\pm \omega \sqrt{x_0^2-x^2}
        \end{aligned}\)}&\multirow{2}{*}[-1em]{\(\begin{aligned}
            a=-\omega^2x
        \end{aligned}\)}& Spring-Mass & Pendulum\\
            &
            &
            \(\begin{aligned}
                T=2\pi\sqrt{\frac{m}{k}}
            \end{aligned}\)&
            \(\begin{aligned}
                T=2\pi\sqrt{\frac{l}{g}}
            \end{aligned}\)
        \\
        \hline
    \end{tabular}
    \item \begin{tabular}{|Sc|Sc|Sc|Sc|}
        \hline
        &
    \(\begin{aligned}
        E_k
    \end{aligned}\)&
    \(\begin{aligned}
        E_p
    \end{aligned}\)&
    \(\begin{aligned}
            E_T
    \end{aligned}\)\\
    \hline
        \(\begin{aligned}
            t
        \end{aligned}\)&
        \(\begin{aligned}
            \frac{1}{2}m\omega^2x_0^2\cos^2(\omega t)
        \end{aligned}\)&
        \(\begin{aligned}
            \frac{1}{2}m\omega^2x_0^2\sin^2(\omega t)
        \end{aligned}\)&
        \(\begin{aligned}
            \frac{1}{2}m\omega^2x_0^2
        \end{aligned}\)\\
        \hline
        \(\begin{aligned}
            m
        \end{aligned}\)&
        \(\begin{aligned}
            \frac{1}{2}m\omega^2(x_0^2-x^2)
        \end{aligned}\)&
        \(\begin{aligned}
            \frac{1}{2}m\omega^2x^2
        \end{aligned}\)&
        \(\begin{aligned}
            \frac{1}{2}m\omega^2x_0^2
        \end{aligned}\)\\
        \hline
    \end{tabular}
    \item Simple pendulums and mass spring systems can be approximated to be SHM when the angle of oscillation (\(\leq 20^\circ\)) and oscillation amplitude are small, respectively.
    \item \begin{tabular}{|Sc|Sc|Sc|Sc|}
        \hline
        & In Phase & Antiphase & Out of Phase\\
        \hline
        \(\Delta \phi\)/rad & 0 & \(\pi\) & nonzero
        \\
        \hline
    \end{tabular}
    \item When damping increases:
    \begin{itemize}[label=\(\circ\)]
        \item Amplitude at \emph{all} frequencies decreases.
        \item (Resonance) frequency at max amplitude shifts gradually to lower frequencies.
        \item Peak (max amplitude) becomes flatter.
    \end{itemize}
\end{itemize}
\chapter{Wave Motion}
\begin{itemize}
    \item A \textit{progressive} wave is a wave in which \emph{energy is carried} from one point to another by means of \emph{vibrations or oscillations} within the wave. Particles within the wave are \emph{not transported along} the wave.
    \item A \textit{phase} is an angle which gives a measure of the \emph{fraction of a cycle} that has been \emph{completed} by an oscillating particle or by a wave.
    \item \emph{Intensity} of a wave is the wave energy incident per unit time per unit area \emph{normal} to the wave.
    \item \textit{Polarisation} of a wave refers to the \emph{confinement} of oscillations in \emph{only} one plane. The plane of oscillations is \emph{parallel} to the direction of energy transfer.  
    \item Malus' Law states that the intensity of a beam of \emph{plane polarised light} after passing through a rotatable polariser is directly proportional to the square of the cosine of the angle through which the polariser is rotated from the position that gives maximum intensity. (\(I=I_{\text{max}}\cos^2(\theta)\))
\end{itemize}
\begin{itemize}[label=\(\square\)]
    \item \begin{tabular}{|Sc|Sc|Sc|}
        \hline
        Phase Difference \(\Delta \phi\) & \(\frac{2\pi}{\lambda}\Delta x\) & \(\frac{2\pi}{T}\Delta t\)\\
        \hline
    \end{tabular}
    \item \begin{tabular}{|Sc|Sc|Sc|Sc|}
            \hline
            \multicolumn{4}{|Sc|}{Intensity}\\
            \hline
            \multirow{2}{*}{Amplitude} & \multicolumn{3}{Sc|}{Wave}\\
            \cline{2-4}
            & Spherical & Circular & Plane\\
            \hline
            \(I \propto A^2\) & \(I \propto \frac{1}{r^2}\) & \(I \propto \frac{1}{r}\) & \begin{minipage}{3cm}
                \vspace{1mm}\begin{center}
                    \(I\) is constant\\
                \tiny (No spreading of waves) \normalsize
                \end{center}
            \end{minipage}\\
            \hline
        \end{tabular}
\end{itemize}

\end{document}