\documentclass[oneside]{book}
\usepackage{../Style/Master}
\usepackage{../Style/boxes}
\usepackage{../Style/DefNoteFact}
\usepackage{../Style/QnsProof}
\usepackage{../Style/Thms}
\usepackage{newpxtext, eulerpx}

\begin{document}

\begin{titlepage}
    \frontmatter
    \begin{tikzpicture}[font=\sffamily,remember picture,overlay]
        \node[fill=Dandelion,anchor=north, minimum width=\paperwidth, minimum height=2cm] (names)
     at ([yshift=-3cm]current page.north) {
      \begin{tabular}{c}
          \\
          \color{blue} {\fontsize{24.88pt}{65pt}\selectfont Document Title}\\[2mm]
          \color{blue} {\fontsize{10pt}{10pt}\selectfont By Author / Whatever}\\
          \\
          \Large \color{blue}{Name}\\
          \\
      \end{tabular}
      };
    \end{tikzpicture}

\begingroup
\let\clearpage\relax
\vspace{5cm}
\tableofcontents
\endgroup
\mainmatter
\end{titlepage}

\raggedright
\setcounter{chapter}{1}
\chapter{Kinematics}
\begin{itemize}
    \item \textit{Distance} is defined as the total length of \emph{path} travelled.
    \item \textit{Velocity} is defined as the rate of change of displacement.
    \item \textit{Acceleration} is defined as the rate of change of velocity.
\end{itemize}
\chapter{Dynamics}
\begin{itemize}
    \item \textit{Newton's First Law of Motion} states that an object at rest will remain at rest and an object in motion will remain in motion at constant velocity in a straight line in the absence of an \emph{external} resultant force.
    \item The \textit{linear momentum} of a body is the product of its mass and velocity. The linear momentum is in the \emph{same direction} as it velocity.
    \item \textit{Newton's Second Law of Motion} states that the rate of change of momentum of a body is directly proportional to the resultant force acting on the body and occurs \emph{in the direction} of the resultant force.
    \item \textit{Newton's Third Law of Motion} states that if body A exerts a force on body B, then body B exerts a force of the \emph{same type} that is equal in magnitude and opposite in direction on body A.
    \item \textit{Impulse} is defined as the product of \emph{average} force acting on an object and the time for which the force acts.
    \item The \textit{Principle of Conservation of Linear Momentum} states that the total momentum of a system remains constant provided no \emph{external} resultant force acts on the system.
\end{itemize}
\chapter{Forces}
\begin{itemize}
    \item \textit{Hooke's Law} states that the force is directly proportional to the extension in a material if its \emph{limit of proportionality} is not exceeded.
    \item The \textit{centre of gravity} of an object is the point at which the entire weight of a body may be considered to act.
    \item The \textit{moment} of a force is equal to the product of the force and the \emph{perpendicular} distance of the \emph{line of action} of the force from the pivot. It is also the turning effect of a force.
    \item \textit{Torque of a couple} is defined as the product of one of the forces and the \emph{perpendicular} distance between the \emph{lines of action} of the forces.
    \item The \textit{Principle of Moments} states that if a body is in equilibrium, the sum of all the clockwise moments about \emph{any axis} must be equal to the sum of anticlockwise moments about the \emph{same axis}.
    \item \textit{Density} is defined as the mass per unit volume of a substance.
    \item \textit{Pressure} is defined as force per unit area, where the force is \emph{acting perpendicularly} to the area.
    \item Deriving \(p=\rho gh\):
    \begin{enumerate}
        \item Consider a point at a depth \(h\) below the surface of a liquid of density \(\rho\). 
        \item The force \(F\) acting perpendicularly on a surface area \(A\) at depth \(h\) is due to the weight of the liquid column above \(A\) to give pressure \(p\). Thus, \(p=\frac{F}{A}=\frac{mg}{A}=\frac{\rho Ah}{g}=\rho gh\).
    \end{enumerate}
    \item \textit{Upthrust} is the upward force exerted by a fluid on a body immersed in the fluid (due to pressure difference in the fluid).
    \item \textit{The origin of upthrust:} Upthrust is a result of the pressure difference between top and bottom surfaces of the body, resulting in a net upwards force being exerted on the body by the third medium in which the body is located.
    \item \textit{Archimedes' Principle} states that when a body is totally or partially immersed in a fluid, it experiences an upward force (upthrust) equal to the weight of fluid displaced.
    \item \textit{The Principle of Floatation} states that, for any object floating in \emph{equilibrium}, the upthrust is equal to the weight of the object.
\end{itemize}
\chapter{Work, Energy, and Power}
\begin{itemize}
    \item \textit{Work done} is defined as the product of a force and the displacement in the direction of the force.
    \item \textit{One joule of work} is defined as the work done by a force of 1 Newton when its \emph{point of application} moves through a distance of 1 metre in the direction of the force.
    \item \textit{Energy} is defined as the ability to do work.
    \item \textit{The Principle of Conservation of Energy} states that energy can neither be created or destroyed in \emph{any process}. It can be transformed from one form to another, and transferred from one body to another.
    \item Deriving \(E_k=\frac{1}{2}mv^2\): 
    \begin{enumerate}
        \item Consider a constant horizontal applied force \(F\) acting on an object of mass \(m\) travelling with initial velocity \(u\) to reach a final velocity \(v\) over a displacement \(s\). 
        \item For uniform acceleration, \(v^2=u^2+2as\) so \(as=\frac{1}{2}(v^2-u^2)\). Combined with Newton's Second Law, \(W=Fs=mas=\frac{1}{2}mv^2-\frac{1}{2}mu^2\). When the object starts from rest, \(u=0\). 
        \item By conservation of energy,\emph{ the work done by force \(F\) must be converted into the kinetic energy \(E_k\) of the object}. Hence, \(E_k=W=\frac{1}{2}mv^2-\frac{1}{2}m(0)^2=\frac{1}{2}mv^2\).
    \end{enumerate}
    \item The \textit{Work-Energy Theorem} states that the net work done by \emph{external} forces acting on a particle is equal to the change in kinetic energy of the particle.
    \item Deriving \(E_p=mgh\):
    \begin{enumerate}
        \item Consider an object from the Earth's surface --- which is taken as the reference for zero gravitational potential energy --- raised up by a \emph{constant force \(F\) equal to and opposite to the weight \(mg\)} of the object such that the object moves up at \emph{constant velocity} to a height \(h_2\). 
        \item Thus, the object moves at constant speed so \(\Delta E_k=0\). Therefore, 
        \begin{align*}
            \Delta E_p&= W\\
            E_p-0&=Fs\\
            E_p&=mgh.
        \end{align*}
        Where \(E_p\) is the gravitational potential energy at height \(h\) above the Earth's surface.
    \end{enumerate}
    \item Know how to \(\Delta E_p=\frac{1}{2}kx^2\) from area under graph.
    \item \textit{Power} is defined as the rate of doing work.
    \item Derive \(P=Fv\): \(P=\frac{\text{d}W}{\text{d}t}=\frac{F\text{d}s}{\text{d}t}=Fv\).
\end{itemize}
\chapter{Temperature and Ideal Gases}
\begin{itemize}
    \item The \textit{Zeroth Law of Thermodynamics} If bodies \(A\) and \(B\) are separately in thermal equilibrium with body \(C\), then bodies \(A\) and \(B\) are in thermal equilibrium with each other.
    \item \textit{One mole} is defined as the amount of substance that contains as many elementary particles as there are atoms in 0.012kg of carbon-12.
    \item \textit{Avogadro's Constant \(N_A\)} is the number of atoms in 0.012kg of carbon-12.
    \item ~\\[-3mm]
    \begin{tabular}{|Sc|Sl|}
        \hline
        & Assumptions of the Kinetic Theory of Gases\\
        \hline
        \textbf{M} & 
        \begin{tabular}{@{}Sl@{}}
            The molecules of the gas are in \emph{rapid} and \emph{random} motion.
          \end{tabular}\\
        \hline
        \textbf{A} & 
        \begin{tabular}{@{}Sl@{}}
            There are \emph{no intermolecular} attractive forces.
        \end{tabular}\\
        \hline
        \textbf{N} & 
        \begin{tabular}{@{}Sl@{}}
        Any gas consists of a \emph{very large number} of molecules.
        \end{tabular}\\
        \hline
        \textbf{T} & 
        \begin{tabular}{@{}Sl@{}}
        The duration of collisions is negligible compared\\ to the time interval between collisions.
        \end{tabular}\\
        \hline
        \textbf{E} & 
        \begin{tabular}{@{}Sl@{}}
        The collisions between gas molecules, and between\\ gas molecules and the container walls are \emph{perfectly elastic}.
        \end{tabular}\\
        \hline
        \textbf{V} & 
        \begin{tabular}{@{}Sl@{}}
        The volume of the gas molecules themselves is negligible \\compared to the volume of the container.
        \end{tabular}\\
        \hline
    \end{tabular}
    \item Deriving \(p=\frac{1}{3}\frac{Nm}{V}\langle c^2 \rangle\):
    \begin{enumerate}
        \item Consider a cubic container of side \(l\) containing \(N\) molecules, each of mass m.
        \item Change in momentum due to \emph{elastic} collision between wall and molecule\(\text{}=2mc_x\)
        \item Time interval between collisions, \(\Delta t=\frac{2l}{c_x}\).
        \item By Newton's 2nd Law, \(F=\frac{2mc_x}{\frac{2l}{c_x}}=\frac{mc_x^2}{l}\).
        \item Since \(A=l^2\), Pressure due to 1 particle, \(p=\frac{mc_x^2}{l^3}=\frac{mc_x^2}{V}\).
        \item Pressure due to \(N\) particles, \(p_N=\frac{Nmc_x^2}{V}\).
        \item By Pythagoras'Theorem, \(c^2=c_x^2+c_y^2+c_z^2\). The average speed in the \(x\), \(y\), and \(z\) directions can be taken to be \(c_x=c_y=c_z\) so \(c^2=3c_x^2\). Now, \(p_N=\frac{Nm\langle \frac{1}{3}c^2\rangle}{V}=\frac{1}{3}\frac{Nm\langle c^2 \rangle}{V}\).
    \end{enumerate}
\end{itemize}
\chapter{First Law of Thermodynamics}
\begin{itemize}
    \item The \textit{heat capacity} of a body is defined as the amount of thermal energy required to raise its temperature by one Kelvin / degree Celsius.
    \item The specific \textit{heat capacity} of a body is defined as the amount of thermal energy required to raise the temperature of one unit mass of the substance by one Kelvin / degree Celsius.
    \item The \textit{specific latent heat} of a body is defined as the thermal energy required to change \emph{phase} of one unit mass of a substance, \emph{without a change in temperature}.
    \item \textit{Internal energy} of a system is a sum of \emph{random distribution} of kinetic and potential energy \emph{associated with the molecules} of the system.
    \item The \textit{First Law of Thermodynamics} states that the \emph{increase} in internal energy of a closed system is the \emph{sum} of heat \emph{supplied} to the system and the work done \emph{on} the system. 
\end{itemize}
\chapter{Gravitational Fields}
\begin{itemize}
    \item \textit{Newton's Law of Gravitation} states that the force of attraction between any two point masses is directly proportional to the product of their masses and inversely proportional to the square of their separation.
    \item A \textit{gravitational field} is a region in space where mass experiences a gravitational force acting on it.
    \item Gravitational field strength at a point is defined as the gravitational force per unit mass acting on a small mass placed at that point
    \item The \textit{gravitational potential energy} of a mass at a point is defined as the work done by an \emph{external agent} in bringing the mass from infinity to that point (without any change in kinetic energy).
    \item \textit{Gravitational potential} at a point is defined as the work done per unit mass by an \emph{external agent} in bringing a mass from infinity to that point (without a change in kinetic energy).
\end{itemize}
\begin{itemize}[label=\(\square\)]
    \item \[
\begin{tikzcd}[row sep=large, column sep=large]
     U_G=-\frac{GMm}{r} \arrow{r}{-\frac{\text{d}}{\text{d}r}} \arrow[swap]{d}{\frac{1}{m}} & F_G=-\frac{GMm}{r^2} \arrow[swap]{d}{\frac{1}{m}} \\
     \phi=-\frac{GM}{r} \arrow{r}{-\frac{\text{d}}{\text{d}r}} & g=-\frac{GM}{r^2}\\
\end{tikzcd}
\]
\item \(U_G=m\phi\) \& \(\Delta U_G=m\Delta\phi\).
\item Gravitational force provides the centripetal force:
\begin{align*}
    F_G&=F_c\\
    \frac{GMm}{r^2}&=mr\omega^2=mr\left(\frac{2\pi}{T}\right)^2\\
    T^2&=\frac{4\pi^2}{GM}r^3\\
    T^2 &\propto r^3
\end{align*}
\item Gravitational force provides the centripetal force:
\begin{alignat*}{2}
    && F_G&=F_c\\
    &\text{For \(A\):}& \hspace{1cm} \frac{Gm_Am_B}{(r_A+r_B)^2}&=m_Ar_A\omega^2\\
    &\text{For \(B\):}& \frac{Gm_Am_B}{(r_A+r_B)^2}&=m_Br_B\omega^2
\end{alignat*}
The centre of mass of the system is at point \(P\) where 
\[m_Ar_A=m_Br_B\]
such that both stars have the same angular velocity \(\omega\).
\item Escape velocity \(v_\text{min}=\sqrt{\frac{2GM}{r}}\) (where Min \(E_k\) needed is the gain in \(E_p\) to reach infinity).
\end{itemize}

\end{document}